\begin{Kur-Description}
	\fontsize{16}{14pt}\selectfont
	\newpage
	
	\part{Введение}
	\label{sec:part}
\section{Введение}
\label{sec:section}
\subsection{Цель работы}
\label{sec:subsection}
 Изучение принципов функционирования черенковских детекторов, их типы, характеристики, области применения. Отдельное внимание уделяется характеристикам прибора, таким как угловое, энергетическое, скоростное и временное разрешение для регистрации электронов с энергиями из диапазона 1 -- 10 МэВ. 

\subsection{Актуальность} 
\label{sec:subsection}
Черенковские детекторы обеспечили возможность постановки и проведения многочисленных экспериментов различных физических направлений, диапазон которых чрезвычайно широк. Они используются при регистрации космических лучей экстремальных энергий (КЛЭЭ). Детекторы занимают особое место в нейтринной астрономии. Сегодня работают нейтринные телескопы на основе черенковских счётчиков (НТ-200, ANTARES, NESTOR) для регистрации нейтрино сверхвысоких энергий. Они позволяют вести спектрометрию частиц, приходящих из космоса. В частности, из спектров электронов определяют их происхождение а также процессы, происходящие на Солнце.

\subsection{Основные этапы работы}
\label{sec:subsection}
Непосредственно из цели вытекают задачи:
\begin{enumerate}
	\item Изучить механизм возникновения черенковского излучения, его свойства, способы его применения;
	\item Классифицировать черенковские счётчики, проанализировать их характеристики, внутреннее устройство;
	\item Собрать информацию о применении черенковских детекторов в физике частиц;
	\item Рассчитать разрешение для дифференциального черенковского счётчика. 
\end{enumerate}
\end{Kur-Description}